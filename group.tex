\documentclass[UTF8]{ctexart}
\usepackage{amsmath}
\usepackage{geometry}

\geometry{a4paper,left=3cm,right=3cm}
\newtheorem{thm}{定理}
\newtheorem{define}{定义}

\title{群论考试试题}
\author{艾鑫}


\begin{document}
\maketitle
\section{有限群:非循环六阶群$G_6^2$}
六阶群有两种结构,其中一个是循环群
\begin{equation}
G_6^1 = \{a,a^2,a^3,a^4,a^5,a^6=e\}
\end{equation}
另外一个就是非循环六阶群$G_6^2 = \{e,a,b,c,d,f\}$,它是最小的非阿贝尔群. 其满足下列关系
\begin{equation}
a^2=b^2=c^2=e, \quad d^2 = f, \quad f^2 = d, \quad fd = df = e
\end{equation}
$G_6^2$的乘法表如下:
\begin{equation}
\begin{array}{|c|c|c|c|c|c|c|}
\hline
G_6^2 & e & a & b & c & d & f \\
\hline
e & e & a & b & c & d & f \\
\hline
a & a & e & d & f & b & c \\
\hline
b & b & f & e & d & c & a \\
\hline
c & c & d & f & e & a & b \\
\hline
d & d & c & a & b & f & e \\
\hline
f & f & b & c & a & e & d \\
\hline
\end{array}
\end{equation}
\subsection{$G_6^2$的构造}

$G_6^2$可以通过保持正三角形不变的所有转动对称变换构成,也就是点群$D_3$.正三角形一共有6个对称操作:
\begin{itemize}
   \item $e$ 恒等变换
   \item $c_3^1$,$c_3^2$分别绕中心点$O$逆时针旋转$2\pi/3$和$4\pi/3$角
   \item $c_{2x}$,$c_{2y}$,$c_{2z}$分别绕$x$,$y$,$z$轴旋转$\pi$角
\end{itemize}
在这种构造中,$c_{2x}$,$c_{2y}$,$c_{2z}$对应于$a$,$b$,$c$;$c_3^1$,$c_3^2$对应于$d$,$f$.

$G_6^2$还可以通过置换群$S_3$来构造。置换群的群元有:
\begin{gather}
\begin{pmatrix}
123 \\
123
\end{pmatrix} = e, \quad
\begin{pmatrix}
123 \\
213
\end{pmatrix} = (12), \quad
\begin{pmatrix}
123 \\
132
\end{pmatrix} = (23) \\
\begin{pmatrix}
123 \\
321
\end{pmatrix} = (31), \quad
\begin{pmatrix}
123 \\
231
\end{pmatrix} = (123), \quad
\begin{pmatrix}
123 \\
312
\end{pmatrix} = (132)
\end{gather}
其中$(12),(23),(31)$对应于$a,b,c$; $(123),(132)$对应于$d,f$.

\subsection{$G_6^2$的子群}
\begin{define}[子群]
群$G$的子集$H$如果在和群$G$相同的乘法规则下也构成群, 则称$H$为群$G$的子群. 
\end{define}

对于任意一个群$G$,群$\{e\}$和群$G$本身一定是$G$的子群. 其他的子群, 叫做真子群或固有子群. 对于六阶非循环群$G_6^2$的真子群有4个:
\begin{equation}
H_1 = \{e,a\}, \quad H_2 = \{e,b\}, \quad H_3 = \{e,c\}, \quad H_4 = \{e,d,f\}
\end{equation}
要判断一个子集$H$是否是子群,关键要看三个地方:一是要看是否有单位元;二是看是否有逆;三是看是否满足封闭性。结合性的满足是自然的,因为$H$是群$G$的子集,群$G$满足结合律,$H$必然满足结合律。很容易验证,上面列出的4个子群都满足上述条件。
\subsection{$G_6^2$的分解}
群$G$的元素可以按照共轭类或者陪集进行分解. 下面给出共轭的定义.
\begin{define}[共轭]
对于群$G$中的两个元素$g_i$,$g_j$,如果存在另一个元素$g\in G$使$g_i = g g_i g^{-1}$成立, 则称$g_i,g_j$是相互共轭的, 用符号$\sim$表示, 记为$g_i \sim g_j$. 
\end{define}

共轭具有下列的性质:
\begin{itemize}
\item 每个元素都与自身共轭, $g_i \sim g_i$; (反身性)
\item 如果$g_i \sim g_j$, 则有$g_j \sim g_i$; (对称性)
\item 如果$g_i \sim g_j$, $g_j \sim g_k$, 则有$g_i \sim g_k$. (传递性)
\end{itemize}

\begin{define}[共轭类]
群$G$内彼此共轭的元素集合构成共轭类,简称类. 
\end{define}

群中的每个元素仅属于一个类, 因为如果一个元素同时属于两个类, 那么由于共轭的传递性, 这两个类就可以合并为一个类. 由于单位元仅与自己共轭, 所以单位元自成一类. 通常用记号$[g]$表示群元$g$所在类的元素集合. 对于$G_6^2$有三个类:
\begin{equation}
[e] = \{e\}, \quad [a] = \{a, b, c\}, \quad [d] = \{d, f\}
\end{equation}
故可以将$G_6^2$按类分解为:
\begin{equation}
G_6^2 = [e] \oplus [a] \oplus [d].
\end{equation}

群还可以按照陪集进行分解, 下面来对陪集进行定义. 
\begin{define}[陪集]
设$H \subset G$为群$G$的子群, 令$g_i \in G, g_i \notin H$, 则集合$g_i H = \{g_i h | h \in H\}$称为子群$H$的左陪集. 类似的可以定义子群$H$的右陪集$H g_i$. 
\end{define}

一般来说, 左陪集不一定和右陪集相等.对于$G_6^2$, 考虑子群$H_1 = \{e, a\}$, 则有
\begin{equation}
b H_1 = \{b, f\}, \quad c H_1 = \{c, d\}, \quad H_1 b = \{b,d\}, \quad H_1 c = \{c, f\}
\end{equation}
如果考虑子群$H_2 = \{e, d, f\}$, 则有
\begin{equation}
a H_2 = H_2 a = \{a, b, c\}
\end{equation}
因此有陪集分解
\begin{equation}
\begin{split}
G_6^2 &= H_1 \oplus b H_1 \oplus c H_1 = H_2 \oplus a H_2 \\
 &= H_1 \oplus H_1 b \oplus H_1 c = H_2 \oplus H_2 a.
\end{split}
\end{equation}
\subsection{$G_6^2$的不变子群}
首先给出不变子群的定义. 
\begin{define}[不变子群]
设$H$是群$G$的一个子群, 如果对任意的$g \in G$都有$gHg^{-1}=H$或$gH = Hg$, 即$H$的每个左陪集和与其对应的右陪集完全相同, 则子群$H$称为群$G$的不变子群或正规子群. 
\end{define}

关于不变子群有如下定理: 
\begin{thm}[不变子群]
如果$H$是群$G$的不变子群,则$H$一定包含群$G$的一些完整的类. 反之, 如果子群$H$包含了群$G$的完整的类, 则$H$一定是群$G$的不变子群. 
\end{thm}

由这个定理知, $G_6^2$的子群中,子群$H_4 = \{e, d, f\}$完整的包含了群$G$的类$\{e\}, \{d, f\}$, 因此$G_6^2$的不变子群为$H_4 = \{e, d, f\}$.

\subsection{商群$G_6^2 / H$}
如果群$H$是群$G$的不变子群, 则可将群$G$分解为下列陪集的直和:
\begin{equation}
G = H \oplus g_1 H \oplus g_2 H \oplus \cdots \oplus g_{l-1}H,
\end{equation}
其中$g_i H = H g_i$. 由此可以定义陪集之间的乘法
\begin{equation}
(g_i H)(g_j H) = g_i H g_j H = g_i g_j H H = g_k H.
\end{equation}
可以证明,这样定义的乘法是自洽的. 这样商集
\begin{equation}
G / H = \{H, g_1 H, g_2 H, \cdots , g_{l-1} H\}
\end{equation}
构成阶数为$l = n / n_H$的群, 其中$n_H$为不变子群$H$的阶, 这个群就称为群$G$的商群.

对于群$G_6^2$, 它有不变子群$H = \{e, d, f\}$, 陪集$M = a H = \{a, b, c\}$,商群为
\begin{equation}
G_6^2 / H = \{H,M\}
\end{equation}

\subsection{$G_6^2$的二维表示矩阵}
首先给出表示的定义:
\begin{define}[群的表示]
如果存在从群$G$到作用在线性向量空间$V$上的算符群$\varGamma_G$的一个同态,即
\begin{equation}
g \in G \mapsto \varGamma_g \in \varGamma_G,
\end{equation}
其中$\varGamma_g$是与群元对应的算符,满足
\begin{equation}
\varGamma_{g_1}\varGamma_{g_2} = \varGamma_{g_1 g_2}
\end{equation}
则算符群$\varGamma_G$称为群$G$的一个表示,线性向量空间$V$的维数称为表示的维数.如果这个同态同时也是同构的, 则该表示称为忠实表示.
\end{define}

如果在$d$维向量空间$V$中选择一组基$\{e_i, i = 1, 2, \cdots, d\}$, 则$\varGamma_g$可以用$d \times d$的矩阵来实现, 具体为
\begin{equation}
\varGamma_g e_i = \sum_{j = 1}^{d} e_j D_{ji} (g) \equiv e_j D_{ji}(g), \quad g \in G, i = 1,2, \cdots, d.
\end{equation}
\subsection{$G_6^2$的不可约表示的正交性}

\subsection{$G_6^2$的特征标的正交性}

\subsection{$G_6^2$的正则表示}

\subsection{$G_6^2$的基础表示}

\subsection{$G_6^2$的特征标表}

\section{李群与李代数:$\mathrm{SO}(3)$}
测试2
\end{document}
